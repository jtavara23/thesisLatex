%!TEX root = origin.TEX
\chapter{Consideraciones finales}
\pagenumbering{arabic}
\setcounter{page}{128}
\renewcommand{\baselinestretch}{2} %doble espacio paratodo el texto
\textheight 21cm

\section{Conclusiones}

	Al finalizar la investigación podemos confirmar que se cumplieron los objetivos específicos planteados.

	Se obtuvo dos datasets de imágenes de señales de tránsito, uno de Alemania distribuidas en 43 categorías y otro de Perú distribuidas en 7 categorías, compuestos inicialmente por 51839 imágenes y 614 imágenes respectivamente.
\vskip 0.2cm
	Posteriormente, se analizaron los datasets y al dividirlos en 3 grupos (entrenamiento, validación y evaluación), se determinó la necesidad de aumentar ambos y esto fue realizado con la ayuda de técnicas de procesamiento de imágenes, resultando en dos datasets robustos de imágenes, el dataset de Alemania compuesto por 283530 imágenes de las cuales 203175 fueron para entrenamiento, 67725 para validación y 12630 para evaluación y el dataset de Perú compuesto por 36012 imágenes de las cuales 23485 fueron para entrenamiento, 3131 para validación y 4698 para evaluación. Ambos datasets ofrecidos para futuras investigaciones.
\vskip 0.2cm
	Fueron diseñados cinco modelos de red para cada dataset, los cuales fueron entrenados, validados y evaluados individualmente.
\vskip 0.2cm
	Luego de experimentar el uso de diversas funciones de activación, funciones de costo, ajuste de parámetros y métodos de optimización, se logró determinar un conjunto de hiperparámetros (Sección 3.2) de los cuales solo la Tasa de Aprendizaje varia para cada dataset de imágenes. 
	\vskip 0.2cm
	Al finalizar el análisis de los 10 modelos (cinco por cada dataset), se puede observar que cuanto más profunda sea la red neuronal, se obtienen mejores resultados. 
\vskip 0.2cm
	De este modo, el modelo final obtenido compuesto principalmente de 4 capas convolucionales, 2 capas totalmente conectadas y un total de 90185 neuronas (Sección 3.2.1.5 - Diseño E - Alemania), contribuye en el reconocimiento de señales de Tránsito de Alemania con una tasa de acierto del {\bf 98.62\%}, mucho mejor que el resultado de 95.29\% obtenido por \citep{Ayuque2016} y mucho más próximo al mejor resultado de 99.46\% obtenido en las investigaciones hechas en base al dataset GTSRB, \citep{Ciresan}.
	\vskip 0.2cm
	
	Para el dataset de señales de tránsito vehicular del Perú, el modelo E con el mismo diseño neuronal y similares configuraciones compuesto por 279623 neuronas (Sección 3.2.1.5 - Diseño E - Perú) permite también obtener un {\bf alto grado de acierto (99.02\%)} tras analizar 4698 imágenes. 
\vskip 0.2cm
	En conclusión, el objetivo general que trata sobre implementar un modelo basado en el aprendizaje profundo de redes neuronales convolucionales para reconocer automáticamente señales de tránsito vehicular fue conseguido a través del Modelo E.	 


\section{Trabajos futuros y recomendaciones}


	El modelo puede ser mejorado(ampliado), teniendo muchas más capas convolucionales y capas totalmente conectadas para poder experimentar si existe o no alguna mejora en los resultados. Además, se recomienda obtener muchas más imágenes para exceder el rendimiento humano, \citep{Goodfellow-et-al-2016}

	El dataset de señales de tránsito del Perú también puede ser ampliado con la finalidad de abarcar más categorías, ya que se tiene confianza por lo mostrado con el dataset de Alemania que el modelo es robusto para soportar mayor cantidad de estas.
	
	Se sugiere integrar el modelo obtenido en un sistema más general que primero localice las señales de tránsito en escenas que abarcan más de una señal de tránsito, para luego proceder a su multireconocimiento.

	%Se recomienda utilizar la herramienta computacional Tensorflow, ya que fue importante en el análisis e implementación de los modelos desarrollados, permitiendo realizar entrenamientos, validaciones y evaluaciones con mucha facilidad.
\newpage