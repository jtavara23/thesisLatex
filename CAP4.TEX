%!TEX root = origin.TEX
\chapter{Materiales y Técnicas}
\pagenumbering{arabic}
\setcounter{page}{68}
\renewcommand{\baselinestretch}{1.2} %doble espacio paratodo el texto

\section{ Tipo de investigación}
	De acuerdo al fin que se persigue es una investigación de tipo tecnológica y de acuerdo al diseño es una investigación experimental de tipo cuantitativa donde se analizará el rendimiento del modelo algorítmico.

\section{Variables de la Investigación}
		
		\subsection{Variable Dependiente}
		\indent Reconocimiento automático de señales de tránsito vehicular
		\subsection {Variable Independiente}
		\indent Modelo basado en el aprendizaje profundo de redes neuronales convolucionales	
		
	\section{Indicadores}
		\renewcommand{\baselinestretch}{2}
		Los indicadores nos permiten realizar mediciones y a su vez determinan la validez de la hipótesis planteada en la presente investigación. El desempeño de la clasificación se puede verificar con los siguinetes cinco indicadores: 
		\begin{table}[H]
		\centering
		\caption{\small{Indicadores para la investigación}}
		\begin{tabular}{|>{\small}c|>{\small}c|>{\small}c|}
		\hline
		{\ul \textbf{Indicador}} & {\ul \textbf{Descripcion}} & {\ul \textbf{Instrumento}} \\ \hline
		\multirow{2}{*}{\begin{tabular}[c]{@{}c@{}}Tasa de Verdaderos Positivos\\ (Efectividad - Sensibilidad)\end{tabular}} & \multirow{2}{*}{$ \frac{Verdaderos\,Positivos}{{Verdaderos\,Positivos}+{Falsos\,Negativos}}$} & \multirow{2}{*}{\begin{tabular}[c]{@{}c@{}}Modelo de\\ Reconocimiento\end{tabular}} \\
		 &  &  \\ \hline
		\multirow{2}{*}{\begin{tabular}[c]{@{}c@{}}Tasa de Verdaderos Negativos \\ (Especificidad)\end{tabular}} & \multirow{2}{*}{$ \frac{Verdaderos\,Negativos}{{Verdaderos\,Negativos}+{Falsos\,Positivos}}$} & \multirow{2}{*}{\begin{tabular}[c]{@{}c@{}}Modelo de\\ Reconocimiento\end{tabular}} \\
		 &  &  \\ \hline
		 \multirow{2}{*}{\begin{tabular}[c]{@{}c@{}}Valor Predictivo Positivo \\ (Precisión)\end{tabular}} & \multirow{2}{*}{$ \frac{Verdaderos\,Positivos}{{Verdaderos\,Positivos}+{Falsos\,Positivos}}$} & \multirow{2}{*}{\begin{tabular}[c]{@{}c@{}}Modelo de\\ Reconocimiento\end{tabular}} \\
		 &  &  \\ \hline
		 \multirow{2}{*}{\begin{tabular}[c]{@{}c@{}}Curvas PR \\ (Precision - Recall)\end{tabular}} & \multirow{2}{*}{Relación entre Efectividad(Recall) y Precision} & \multirow{2}{*}{\begin{tabular}[c]{@{}c@{}}Modelo de\\ Reconocimiento\end{tabular}} \\
		 &  &  \\ \hline
		%\multirow{2}{*}{\begin{tabular}[c]{@{}c@{}}Tiempo de\\ Procesamiento\end{tabular}} & \multirow{2}{*}{$S= \frac{\sum_{i=1}^{Total} tiempo_{i}}{Total\,de\,muestras}$} & \multirow{2}{*}{\begin{tabular}[c]{@{}c@{}}Cronometro\\ Computacional\end{tabular}} \\
		\multirow{2}{*}{\begin{tabular}[c]{@{}c@{}}Curvas ROC \\ (Receiver Operating Characteristic)\end{tabular}} & \multirow{2}{*}{Relación entre Efectividad y Especificidad} & \multirow{2}{*}{\begin{tabular}[c]{@{}c@{}}Modelo de\\ Reconocimiento\end{tabular}} \\
		 &  &  \\ \hline
		\multirow{2}{*}{\begin{tabular}[c]{@{}c@{}}Tasa de Acuracia \\ (Exactitud)\end{tabular}} & \multirow{2}{*}{$ \frac{Verdaderos\,Positivos+Verdaderos\,Negativos}{Total\,de\,Imagenes}$} & \multirow{2}{*}{\begin{tabular}[c]{@{}c@{}}Modelo de\\ Reconocimiento\end{tabular}} \\
		 &  &  \\ \hline		 
		\end{tabular}
		\vspace{-1.0em}
		\end{table}

	\newpage
	\section{Recolección de Datos para la Construcción del Modelo}
		\renewcommand{\baselinestretch}{1.2}
		\subsection{Técnica de Recolección}
		\begin{enumerate}		
			\item[]   {Revisión de la literatura(análisis de documentos)}
		\end{enumerate}

		\subsection{Población}
		\begin{enumerate}		
			\item[] Existen diversas arquitecturas de redes neuronales convolucionales usadas para propósitos específicos. La idea principal es que al principio la arquitectura de la red neuronal toma como entrada una imagen y su especificación de dimensiones en 3 valores(largo, ancho y profundidad). Capas convolucionales y capas de activación(optimización) se apilan juntas y luego son seguidas por capas de agrupamientos. Esta estructura se usa comúnmente y se repite hasta que la entrada (imagen) se fusiona espacialmente a un tamaño pequeño. Después de eso, se envía a capas completamente conectadas y la salida de la última capa completamente conectada, que está al final de la arquitectura, produce los puntajes de clase de la imagen de entrada.
		\end{enumerate}
		
	
		\subsection{Muestra}
		\begin{enumerate}		
		\item[] Para el proceso de diseño e implementación de arquitecturas a elaborar se tomarán en cuenta dos modelos mundialmente conocidos e importantes:
		\end{enumerate}
				
		
		\subsubsection{Arquitectura AlexNet} 
			Esta arquitectura hizo que las Redes Convolucionales fueran populares en el campo de Visión por Computadora. AlexNet fue desarrollado por \citep{Krizhevsky2012}. La entrada consiste en una imagen de 224x224 pixeles en formato RGB (3 canales). La arquitectura consta de 8 capas, las primeras cinco capas son convolucionales y el resto son capas totalmente conectadas. La primera y segunda capa convolucional son seguidas por capas de normalización de respuesta local, luego estas capas de normalización de respuesta son seguidas por capas de agrupación máxima. La salida de cada capa convolucional y cada capa completamente conectada se activa a través de  una función no linear conocida como RELU. Similar a LenNet-5\citep{LeCun} pero más grande, teniendo aproximadamente 60 millones de parámetros.

		\subsubsection{Arquitectura Inception}  
			Es una arquitectura de red neuronal convolucional profunda, creada por un grupo de investigación de Google que fue responsable de establecer el nuevo estado del arte de la técnica para la clasificación y detección en la competencia de reconocimiento visual a gran escala ImageNet 2014(ILSVRC14). 
			\vskip 0.1cm
			El principal sello distintivo de esta arquitectura es la utilización mejorada de los recursos informáticos dentro de la red. Esto fue logrado por un cuidadoso que permite aumentar la profundidad y el ancho de la red mientras se manteniene el costo computacional constante. Para optimizar la calidad, las decisiones para elaborar la arquitectura se basaron en el principio Hebbiano y la intuición de procesamiento a escala múltiple. Una encarnación particular utilizada en la competencia ILSVRC14 es llamada GoogLeNet, una red de 22 capas de profundidad, cuya calidad se evalúa en el contexto de reconocimiento y detección, \citep{Inception}.

		
	\section{Recolección de Datos para el Entrenamiento y Evaluación del Modelo}
		\subsection{Técnica de Recolección}
		\begin{enumerate}		
			\item[]  Revisión de la literatura(análisis de documentos) y captura de imágenes del Perú a través de la aplicación Google Maps
		\end{enumerate}

		\subsection{Población} 
		\begin{enumerate}
		\item[]				
		{\bf *) Área:} Imágenes de Señales de Seguridad Vial.\vskip 0.1cm
		{\bf *) Categoría:} Tránsito Vehicular Vertical.\vskip 0.1cm
		{\bf *) Subcategoría:} Señales reguladoras, preventivas e informativas.\vskip 0.1cm
		\end{enumerate}
		\begin{enumerate}		
			\item[]  La población es infinita para esta investigación, debido al número infinito de formas distintas en que una señal de tránsito puede ser capturada.Se tomaran en cuenta imágenes donde se muestren señales de tránsito vehicular del tipo vertical en sus 3 subcategorias.
		\end{enumerate}

		\subsection{Muestra} 
		\begin{enumerate}		
			\item[]	Existe una colección(dataset) que se ajusta a las características de nuestra población y será usada para conseguir el objetivo de la investigación, principalmente porque cuenta con abundantes imágenes lo que conforma una muestra representativa y necesaria para hacer generalizaciones. 
		\end{enumerate}

		\subsubsection{Señales de Tránsito de Alemania} \citep{Stallkamp-IJCNN-2011} 
			Conjunto de datos creados a partir de aproximadamente 10 horas de video grabados durante el día mientras se conducía en diferentes tipos de carreteras en Alemania. De las secuencias del video fueron extraídas imágenes de señales de tránsito en formato RGB cuyas dimensiones varían entre 15x15 y 250x250 pixeles. En esta colección se obtuvieron un total de 39209 imágenes distribuidas en 43 clases. 
		
		%\item[B)] {\bf Señales de Tránsito de Bélgica} \citep{Timofte-MVA-2011} \newline
		%	Conjunto de datos creados con a la captura simultánea de imágenes durante el día realizada por 8 cámaras mientras se conducía aproximadamente a 35Km/h, con el objetivo de obtener una localización 3D y un refinamiento simultáneo. Las señales de tránsito fueron capturados a una distancia de menos de 50 metros y almacenadas en formato PPM cuyas dimensiones varían entre 25x30 y 260x250 pixeles. En esta colección se obtuvieron un total de 7125 imágenes distribuidas en 62 clases.
		
		
		\subsubsection{Señales de Tránsito de Perú} 
			Conjunto de aproximadamente 614 imágenes originalmente tomadas de la aplicación Google Maps agrupadas en 7 categorías. Dado que la población de imágenes es infinita, se optó por usar un muestreo aleatorio simple para poblaciones desconocidas con un nivel de confianza del 95\% y un error de muestreo del 4\%.
		
			\vskip 0.4cm
			\begingroup\makeatletter\def\f@size{17.8}\check@mathfonts
			\begin{center}
				${\bf n} ={\bf \frac{z^2pq}{e^2}}$
			\end{center}
			\endgroup
		
			En el que:\vskip 0.1cm
			\begin{itemize}
				\item $n$ = tamaño de muestra
				\item $z$ = Coeficiente de confiabilidad 95\% al que corresponde (1.96)
				\item $pq$ = Varianza de la población, ponemos la varianza mayor posible porque a mayor varianza hará falta una muestra mayor(0.25)
				\item $e$ = Error muestral(0.04)
			\end{itemize}
