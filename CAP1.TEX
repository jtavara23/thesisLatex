%!TEX root = origin.TEX
%%%%%%%%%%%%%%%%%%%%%%%%%%%%% CARATULA%%%%%%%%%%%%%%%%%%%%%%%
  \textheight 19cm
  \pagestyle{empty}
  \begin{center}
   {\bf {\fontsize{14}{16.8}\selectfont UNIVERSIDAD NACIONAL DE TRUJILLO}}     
   
      {\bf{\fontsize{14}{16.8}\selectfont Facultad de Ciencias Físicas y Matemáticas}} 

    {\bf{\fontsize{14}{16.8}\selectfont Escuela Profesional de Informática}}
  \end{center}  

  \begin{figure}[ht]
  \begin{center}
  \includegraphics[width=.4\textwidth]{images/unt}
  \end{center}
  \end{figure}

  \vskip 1.15cm
  \begin{center}
    {  
      {\bf \fontsize{17}{20.4}\selectfont{MODELO DE RECONOCIMIENTO AUTOMÁTICO DE SEÑALES DE TRÁNSITO VEHICULAR MEDIANTE APRENDIZAJE PROFUNDO DE REDES NEURONALES CONVOLUCIONALES }}     
      \vskip 3cm
      {\bf \fontsize{14}{16.8}\selectfont {\hspace{-0.4cm}AUTOR: Josué Gastón Távara Idrogo}}\\
      \vskip 0.2cm
      {\bf \fontsize{14}{20.4}\selectfont{\hspace*{-1.7cm} ASESOR: Jorge Luis Gutierrez Gutierrez }}  
    } 
  \end{center}   

  \vskip 1.1cm
  \begin{center}    
  {\bf {\fontsize{14}{16.8}\selectfont Trujillo - Perú
  \vskip 0.0cm
  \hspace*{-0.2cm} 
  2019 }}
  \end{center} 
  \newpage


%%%%%%%%%%%%%%%%%%%%%%%%%%%%CONTRA CARATULA 1 %%%%%%%%%%%%%%%%%%%%%%%%%%%%%
  \newpage
  \pagestyle{plain}
  \pagenumbering{roman}

  \hspace*{6cm}
  \vskip 9cm
  \begin{center}
     {\bf \doublespacing {\fontsize{17}{20.4}\selectfont{MODELO DE RECONOCIMIENTO AUTOMÁTICO DE SEÑALES DE TRÁNSITO VEHICULAR MEDIANTE APRENDIZAJE PROFUNDO DE REDES NEURONALES CONVOLUCIONALES }}}     
  \end{center} 
  \newpage


%%%%%%%%%%%%%%%%%%%%%%%%%%%%% CONTRA CARATULA 2 %%%%%%%%%%%%%%%%%%%%%%%
  \begin{center}
   {\bf {\fontsize{14}{16.8}\selectfont{JOSUÉ GASTÓN TÁVARA IDROGO}}}
   \end{center}   

  \vskip 3.2cm

  \begin{center}
     {\bf \doublespacing {\fontsize{17}{20.4}\selectfont{MODELO DE RECONOCIMIENTO AUTOMÁTICO DE SEÑALES DE TRÁNSITO VEHICULAR MEDIANTE APRENDIZAJE PROFUNDO DE REDES NEURONALES CONVOLUCIONALES }}}     
  \end{center}   
    \vskip 2cm
  \begin{verse}
   \fontsize{12}{14.4}\selectfont{\hspace*{0.6cm}Tesis presentada a la Escuela Profesional de Informática en la Facultad de Ciencias Físicas y Matemáticas de la Universidad Nacional de Trujillo, como requisito para la obtención del Título profesional de Ing. Informático}
  \end{verse}

  \vskip 1.5cm 
  {\fontsize{14}{16.8}\selectfont ASESOR: DR. JORGE LUIS GUTIERREZ GUTIERREZ} 
  \vskip 1cm 
  
  \begin{center}    
  \vskip 1.8cm
  {
  \fontsize{14}{16.8}\selectfont Trujillo - Perú
  \vskip 0.2cm  \hspace*{-0.2cm} 2019
  }
  \end{center} 
  \newpage


%%%%%%%%%%%%%%%%%%%%%%%%%%%%HOJA DE APROBACION %%%%%%%%%%%%%%%%%%%%%%%%%%%%%
\begin{center}
 {\bf {\Large HOJA DE APROBACIÓN }     
 \vskip 1.5cm
  {\Large Modelo De Reconocimiento Automático De Señales De Tránsito Vehicular Mediante Aprendizaje Profundo De Redes Neuronales Convolucionales }}
 \vskip 1cm 
  {\large{Josué Gastón Távara Idrogo}}\\

 \vskip 1cm
\end{center} 
Tesis defendida y aprobada por el jurado examinador:
\vskip 1.2 cm
\begin{flushleft} 
$\overline{\mbox{Prof. Dr. Jorge Luis Gutierrez Gutierrez - Asesor}}$\\
\vskip -0.5cm
Departamento de Informática - UNT
\end{flushleft} 
\vskip 1cm
\begin{flushleft} 
$\overline{\mbox{Prof. Mg. Anthony Gomez Gonzales}}$\\
\vskip -0.5cm
Departamento de Informática - UNT
\end{flushleft} 
\vskip 1cm
\begin{flushleft} 
$\overline{\mbox{Prof. Mg. Jose Luis Peralta Luján}}$\\
\vskip -0.5cm
Departamento de Informática - UNT
\end{flushleft}
\vskip 0.5cm 
\begin{center}    
Trujillo, DIA de MES del 2019
\end{center} 
\newpage
%%%%%%%%%%%%%%%%%%%%%%%%%%%%%%%%%%%%%%%%%%%%%%%%%%%%%%%%%%%%%%%%%%%%%%%%%%%%


%%%%%%%%%%%%%%%%%%%%%%%%%%%% DEDICATORIA %%%%%%%%%%%%%%%%%%%%%%
 
 \addcontentsline{toc}{chapter}{Dedicatoria}
 {\bf\Large {Dedico esta tesis a :}}
 \vskip 1cm
  \begin{quotation}
  {\it Mi madre María y mi padre Gastón; amare et sapere vix deo concesitur.
    \vskip 1cm
    Mis hermanas Carol y Marita por el apoyo continuo.
  }
 \end{quotation}


%%%%%%%%%%%%%%%%%%%%%%%%%%%% AGRADECIMENTOS %%%%%%%%%%%%%%%%%%%%%%
  \newpage
  \addcontentsline{toc}{chapter}{Agradecimientos}
  {\bf\Large {\flushleft{Agradecimientos}}}
  \vskip 1.5cm
  \begin{quotation}
  Agradezco a Dios por haberme bendecido en toda mi vida.
  \vskip 1cm
  A mis profesores del Departamento de Informática, de los cuales recibí una gran cantidad de conocimientos y apoyo que formaron parte fundamental en el desarrollo de mi carrera universitaria.
  \vskip 1cm
  A mi asesor Dr. Jorge Luis Gutierrez Gutierrez de la Universidad Nacional de Trujillo, que siempre se mostró disponible, interesado y capacitado para ayudarme, otorgándome las sugerencias necesarias para redactar esta investigación.
  \vskip 1cm
  A la profesora Dra. Roseli Aparecida Francelin Romero de la Universidad de Sao Paulo por haber introducido el tema de Deep Learning en mi instancia como estudiante de esa universidad y haberme sugerido temas de investigación en esta área.

  \vskip 1cm
  \end{quotation}


%%%%%%%%%%%%%%%%%%%%%%%%%%%% RESUMEN%%%%%%%%%%%%%%%%%%%%%%
  \newpage
  \begin{center}
   \addcontentsline{toc}{chapter}{Resumen}
   {\bf\LARGE Resumen}
  \end{center} 
  \vskip 0.5cm
  \begin{quotation}
  
  La presente investigación tiene por objetivo principal implementar un modelo basado en el aprendizaje profundo de redes neuronales convolucionales para reconocer automáticamente señales de tránsito vehicular usando fundamentos de cálculo matemático, técnicas de procesamiento de imágenes y algoritmos de inteligencia artificial.
  \vskip 0.2cm
  El proyecto se centra en un grupo de señales de Tránsito vehicular de Alemania y Perú, identificando 43 y 7 categorías respectivamente. Iniciando con  la adquisición de imágenes, se procedió a realizar el procesamiento de estas con la finalidad de aumentar el conjunto de datos y poder ejecutar el aprendizaje profundo a través de diversos diseños de arquitecturas de redes neuronales convolucionales.
  \vskip 0.2cm
  Como resultado final, se obtuvo un modelo con buenos indicadores y resultados en el reconocimiento de señales de tránsito vehicular. De esta manera, se pretende contribuir en los esfuerzos de la industria automotriz en el campo de sistemas avanzados de asistencia al conductor, así como también puede formar parte de diversos mecanismos que buscan dar soluciones a la inseguridad vial.

  \vskip 0.2cm
  {\bf Palabras claves:} aprendizaje profundo, redes neuronales convolucionales, procesamiento de imágenes.
  \end{quotation}


%%%%%%%%%%%%%%%%%%%%%%%%%%%%ABSTRACT%%%%%%%%%%%%%%%%%%%%%%
  \newpage
  \begin{center}
   \addcontentsline{toc}{chapter}{Abstract}
   {\bf\LARGE Abstract}\vskip 1.5cm
  \end{center} 
  \begin{quotation}

  The main objective of this research is to implement a model based on the deep learning through convolutional neural networks to automatically recognize vehicular traffic signals using mathematical calculation funds, image processing techniques and artificial intelligence algorithms.
  \vskip 0.2cm
  
  The project focuses on a group of traffic signals from Germany and Peru, identifying 43 and 7 categories respectively. Starting with the acquisition of images, the processing of these activities is carried out with the purpose of increasing the data set so then be able to carry out in-depth learning through various designs of convolutional neural network architectures.
  \vskip 0.2cm

  As a final result, a model with good indicators and results in the recognition of vehicular traffic signals was obtained. In this way, it is intended to contribute to the efforts of the automotive industry in the field of Advanced driver-assistance systems(ADAS), as well as being part of various mechanisms that seek to provide solutions to road safety.
  \vskip 0.3cm
  \hspace*{-0.6cm}{\bf Keywords:} deep learning, convolutional neural networks, image processing.
  \end {quotation}



%%%%%%%%%%%%%%%%%%%%%%%%%%% LISTA DE SIMBOLOS %%%%%%%%%%%%%%%%%%%%%%
%\newpage
%\addcontentsline{toc}{chapter}{Lista de símbolos}
% {\bf\LARGE Lista de símbolos(FALTA POR PRECISAR)}
% \vskip 1.5cm
%Constantes: 
%\begin{enumerate}
%\item[(1)]$\otimes$ \hspace*{0.8cm}Simbolo de convolución}
%\item[(2)] $n $ \hspace*{1.1cm} Indice de bienes finales deseados por los consumidores.
%\item[(3)] ...
%%\vskip 3cm
%\end{enumerate} 
%%\vskip 0.3cm
%Variables:
%\begin{enumerate}
%\item[(5)] $ x^{r} $ \hspace*{1cm} Vector columna que denota la actividad de producción.
%\item[(6)] $ u^{r} $ \hspace*{1.2cm} . . .
%\end{enumerate}
