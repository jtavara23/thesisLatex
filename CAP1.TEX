%!TEX root = origin.TEX
%%%%%%%%%%%%%%%%%%%%%%%%%%%%% CARATULA%%%%%%%%%%%%%%%%%%%%%%%
  \textheight 19cm
  \pagestyle{empty}
  \begin{center}
   {\bf {\fontsize{14}{16.8}\selectfont UNIVERSIDAD NACIONAL DE TRUJILLO}}     
   
      {\bf{\fontsize{14}{16.8}\selectfont Facultad de Ciencias Físicas y Matemáticas}} 

    {\bf{\fontsize{14}{16.8}\selectfont Escuela Académico Profesional de Informática}}
  \end{center}  

  \begin{figure}[ht]
  \begin{center}
  \includegraphics[width=.4\textwidth]{images/unt}
  \end{center}
  \end{figure}

  \vskip 1.15cm
  \begin{center}
    {  
      {\bf \fontsize{17}{20.4}\selectfont{MODELO DE RECONOCIMIENTO AUTOMÁTICO DE SEÑALES DE TRÁNSITO VEHICULAR MEDIANTE APRENDIZAJE PROFUNDO DE REDES NEURONALES CONVOLUCIONALES }}     
      \vskip 2.5cm
      {\fontsize{17}{20.4}\selectfont {\bf Autor:} Josué Gastón Távara Idrogo}
      \vskip 1cm
      {\fontsize{17}{20.4}\selectfont{\hspace*{-1cm}  {\bf Asesor:} Dr. Jorge Luis Gutierrez Gutierrez }}  
    } 
  \end{center}   

  \vskip 1.0cm
  \begin{center}    
  {\bf {\fontsize{14}{16.8}\selectfont Trujillo - La Libertad
  \vskip 0.5cm
  \hspace*{-0.1cm} 
  2018 }}
  \end{center} 
  \newpage


%%%%%%%%%%%%%%%%%%%%%%%%%%%%CONTRA CARATULA 1 %%%%%%%%%%%%%%%%%%%%%%%%%%%%%
  \newpage
  \pagestyle{plain}
  \pagenumbering{roman}

  \hspace*{6cm}
  \vskip 9cm
  \begin{center}
     {\bf \doublespacing {\fontsize{17}{20.4}\selectfont{MODELO DE RECONOCIMIENTO AUTOMÁTICO DE SEÑALES DE TRÁNSITO VEHICULAR MEDIANTE APRENDIZAJE PROFUNDO DE REDES NEURONALES CONVOLUCIONALES }}}     
  \end{center} 
  \newpage


%%%%%%%%%%%%%%%%%%%%%%%%%%%%% CONTRA CARATULA 2 %%%%%%%%%%%%%%%%%%%%%%%
  \vskip 10cm 
  \begin{center}
     {\bf \doublespacing {\fontsize{17}{20.4}\selectfont{MODELO DE RECONOCIMIENTO AUTOMÁTICO DE SEÑALES DE TRÁNSITO VEHICULAR MEDIANTE APRENDIZAJE PROFUNDO DE REDES NEURONALES CONVOLUCIONALES }}}     
  \end{center}   
    \vskip 2cm
  \begin{verse}
   \fontsize{12}{14.4}\selectfont{\hspace*{0.6cm}Tesis presentada a la Escuela Académico Profesional de Informática en la Facultad de Ciencias Físicas y Matemáticas de la Universidad Nacional de Trujillo, como requisito parcial para la obtención del grado de Bachiller en ciencia de la computación ( Título profesional de Ing. Informático)}
  \end{verse}

  \vskip 1.5cm 
  {\fontsize{14}{16.8}\selectfont AUTOR: JOSUÉ GASTÓN TÁVARA IDROGO} 
  \vskip 1.5cm 
  {\fontsize{14}{16.8}\selectfont ASESOR: JORGE LUIS GUTIERREZ GUTIERREZ} 
  \vskip 1cm 
  
  \begin{center}    
  \vskip 1.3cm
  {
  \fontsize{14}{16.8}\selectfont Trujillo - La Libertad
  \vskip 0.2cm  \hspace*{-0.2cm} 2018
  }
  \end{center} 
  \newpage


%%%%%%%%%%%%%%%%%%%%%%%%%%%%HOJA DE APROBACION %%%%%%%%%%%%%%%%%%%%%%%%%%%%%
\begin{center}
 {\bf {\Large HOJA DE APROBACIÓN }     
 \vskip 1.5cm
  {\Large Modelo De Reconocimiento Automático De Señales De Tránsito Vehicular Mediante Aprendizaje Profundo De Redes Neuronales Convolucionales }}
 \vskip 1cm 
  {\large{Josué Gastón Távara Idrogo}}\\

 \vskip 1cm
\end{center} 
Tesis defendida y aprobada por el jurado examinador:
\vskip 1.2 cm
\begin{flushleft} 
$\overline{\mbox{Prof. Dr. Jorge Luis Gutierrez Gutierrez - Asesor}}$\\
\vskip -0.5cm
Departamento de Informática - UNT
\end{flushleft} 
\vskip 1cm
\begin{flushleft} 
$\overline{\mbox{Prof. Mg. Anthony Gomez Gonzales}}$\\
\vskip -0.5cm
Departamento de Informática - UNT
\end{flushleft} 
\vskip 1cm
\begin{flushleft} 
$\overline{\mbox{Prof. Mg. Jose Luis Peralta Luján}}$\\
\vskip -0.5cm
Departamento de Informática - UNT
\end{flushleft}
\vskip 0.1cm 
\begin{center}    
Trujillo, 23 de octubre del 2018
\end{center} 
\newpage
%%%%%%%%%%%%%%%%%%%%%%%%%%%%%%%%%%%%%%%%%%%%%%%%%%%%%%%%%%%%%%%%%%%%%%%%%%%%


%%%%%%%%%%%%%%%%%%%%%%%%%%%% DEDICATORIA %%%%%%%%%%%%%%%%%%%%%%
 
 \addcontentsline{toc}{chapter}{Dedicatoria}
 {\bf\Large {Dedico esta tesis a :}}
 \vskip 1cm
  \begin{quotation}
  {\it Mi madre y mi padre; amare et sapere vix deo concesitur.
    \vskip 1cm
    Mis hemanas por el apoyo continuo y paciencia.
  }
 \end{quotation}


%%%%%%%%%%%%%%%%%%%%%%%%%%%% AGRADECIMENTOS %%%%%%%%%%%%%%%%%%%%%%
  \newpage
  \addcontentsline{toc}{chapter}{Agradecimientos}
  {\bf\Large {\flushleft{Agradecimientos}}}
  \vskip 1.5cm
  \begin{quotation}
  Agradezco a Dios por haberme bendecido en toda mi vida.
  \vskip 1cm
  A mis profesores del Departamento de Informática, de los cuales recibi una gran cantidad de conocimientos  . . .
  \vskip 1cm
  A mi asesor Prof. Dr. Jorge Luis Gutierrez Gutierrez que siempre se mostro disponible, interesado y capacitado para ayudarme, otorgándome las sugerencias necesarias para redactar esta investigación.
  \vskip 1cm
  A la profesora Dra. Roseli Aparecida Francelin Romero de la Universidad de Sao Paulo por haber introducido el tema de Deep Learning en mi instancia como estudiante y haberme sugerido temas de investigación en esta área.

  \vskip 1cm
  \end{quotation}


%%%%%%%%%%%%%%%%%%%%%%%%%%%% RESUMEN%%%%%%%%%%%%%%%%%%%%%%
  \newpage
  \begin{center}
   \addcontentsline{toc}{chapter}{Resumen}
   {\bf\LARGE Resumen}
  \end{center} 
  \vskip 0.5cm
  \begin{quotation}
  
  La presente investigación tiene por objetivo principal implementar un modelo basado en el aprendizaje profundo de redes neuronales convolucionales para reconocer automáticamente señales de tránsito vehicular, usando fundamentos de cálculo matemático, técnicas de procesamiento de imágenes y algoritmos de inteligencia artificial.
  \vskip 0.2cm
  Esta investigación pretende contribuir en la industria automotriz, especificamente en los campos de construcción de vehículos autónomos y de los sistemas avanzados de asistencia al conductor, iniciando con la adquisición de imágenes, luego para el pre-procesamiento, se implementará algoritmos de realce de contraste, reducción de ruido, rotación y proyecciones de escalamiento con la finalidad de aumentar el conjunto de datos y poder ejecutar el aprendizaje profundo a través de arquitecturas de redes neuronales convolucionales. Se van a realizar diferentes diseños de arquitecturas convolucionales y se escogerá el que obtenga los mejores resultados.


  \vskip 0.3cm
  \hspace*{-0.6cm}{\bf Palabras claves:} aprendizaje profundo, redes neuronales convolucionales, procesamiento de imágenes.
  \end{quotation}


%%%%%%%%%%%%%%%%%%%%%%%%%%%%ABSTRACT%%%%%%%%%%%%%%%%%%%%%%
  \newpage
  \begin{center}
   \addcontentsline{toc}{chapter}{Abstract}
   {\bf\LARGE Abstract}\vskip 1.5cm
  \end{center} 
  \begin{quotation}

  The main objective of this research is to implement a model based on deep learning of convolutional neural networks to automatically recognize traffic signals, using fundamentals of mathematical calculation, image processing techniques and artificial intelligence algorithms.
  \vskip 0.2cm
  This research aims to contribute in the automotive industry, specifically in the fields of autonomous vehicle construction and advanced driver assistance systems, starting with the acquisition of images, then for the previous processing, the algorithm of contrast reality, reduction is implemented of noise, rotation and scaling projections in order to increase the data set and be able to execute deep learning through convolutional neural network architectures. Different designs of convolutional architectures will be made and the one that obtains the best results will be chosen.

  \vskip 0.3cm
  \hspace*{-0.6cm}{\bf Keywords:} deep learning, convolutional neural networks, image processing.
  \end {quotation}



%%%%%%%%%%%%%%%%%%%%%%%%%%% LISTA DE SIMBOLOS %%%%%%%%%%%%%%%%%%%%%%
%\newpage
%\addcontentsline{toc}{chapter}{Lista de símbolos}
% {\bf\LARGE Lista de símbolos(FALTA POR PRECISAR)}
% \vskip 1.5cm
%Constantes: 
%\begin{enumerate}
%\item[(1)]$\otimes$ \hspace*{0.8cm}Simbolo de convolución}
%\item[(2)] $n $ \hspace*{1.1cm} Indice de bienes finales deseados por los consumidores.
%\item[(3)] ...
%%\vskip 3cm
%\end{enumerate} 
%%\vskip 0.3cm
%Variables:
%\begin{enumerate}
%\item[(5)] $ x^{r} $ \hspace*{1cm} Vector columna que denota la actividad de producción.
%\item[(6)] $ u^{r} $ \hspace*{1.2cm} . . .
%\end{enumerate}
