%!TEX root = origin.TEX
\chapter{Consideraciones finales}
\pagenumbering{arabic}
\setcounter{page}{24}
\renewcommand{\baselinestretch}{2} %doble espacio paratodo el texto


\section{Conclusiones}

Conclusion
Our MCDNN won the German traffic sign recognition benchmark with
a  recognition  rate  of  99.46%,  better  than  the  one  of  humans  on  this  task
(98.84%), with three times fewer mistakes than the second best competing
algorithm  (98.31%).   Forming  a  MCDNN  from  25  nets,  5  per  preprocess-
ing  method,  increases  the  recognition  rate  from  an  average  of  98.52%  to
99.46%.  None of the preprocessing methods are superior in terms of single
DNN  recognition  rates,  but  combining  them  into  a  MCDNN  increases  ro-
bustness to various types of noise and leads to more recognized traffic signs.
We plan to embed our method in a more general system that first localizes
traffic signs in realistic scenes and then classifies them


{\bf Ejemplo}\\
La investigación bibliográfica revela que realmente existe una preocupación de los gobiernos federales con el destino final de los residuos sólidos, con el objetivo de preservar la salud de la población y el medio ambiente urbano y rural. Por ejemplo, se observa la creación, en el caso del Brasil, de la Ley 12.305/10. Sin embargo existe una laguna entre las metas propuestas en la ley con las metas reales de los gobiernos locales. Eso se debe a la falta de una buena estructura organizacional, gerencial y operacional de los gobiernos locales capaz de atender las demandas locales y las necesidades de la población.
\vskip 0.3cm
La falta de cuadros especializados, tanto en los gobiernos centrales como locales, para realizar la planificación y modelamiento de una red logística reversa puede ser compensada con la contribución de los investigadores que actúan en ese campo del conocimiento. Es muy difícil la formación de un equipo que tenga todo el conocimiento en las áreas de ciencia de la computación, de geo procesamiento, de modelamiento matemático y de logística reversa, entre otras. Esa es una de las principales justificativas que los gobiernos, federales y locales, argumentan a la falta de planificación de una red logística reversa que funciones eficaz y eficientemente. 
\vskip 0.3cm
Por lo tanto, como quedó demostrado a lo largo de este trabajo, es posible realizar el modelamiento matemático para este tipo de problema con baja inversión, así como aplicarlo en varias regiones sin necesidad de grandes cambios en el modelamiento propuesto. El modelo propuesto calcula los flujos en la red logística reversa, permitiendo dimensionar la cantidad y capacidad de las unidades productivas y de los vehículos. 
\vskip 0.3cm
...


\section{Trabajos futuros}

